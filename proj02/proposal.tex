%!TEX TS-program = pdflatex or xelatex
%!TEX encoding = UTF-8 Unicode

\documentclass{article}
\usepackage[letterpaper, margin=1in, left=0.75in, right=0.75in] {geometry}
\usepackage{microtype} % micro appearance
\usepackage[all]{nowidow} % no lone lines
\usepackage{changepage} % changes layout mid-page
\usepackage{enumitem} % enum (itemize, enumerate, description)
\usepackage{ifthen}
\usepackage{xfrac}
\usepackage{fancyhdr} % headers
\usepackage{amsmath} % matrices
\usepackage{amssymb} % symbols
\usepackage{relsize}
\usepackage{gensymb} % symbols
\usepackage{tikz} % graphics?
\usepackage{bm} % bold math
\usepackage{multicol} % multiple columns
\usepackage{lipsum} % lipsum

\pagestyle{fancy}

\newcommand {\classname} {CS 4740/5740 -- Intro to Natural Language Processing}
\newcommand {\duedate} {Thu, 2014--03--06}
\newcommand {\hwtype} {Project}
\newcommand {\hwnum} {2: Word Sense Dab.\ (proposal)}
\newcommand {\innermargin} {0.15in} % indentation amt of nested lst

\let\tss\textsuperscript
\let\oldunderline\underline
\renewcommand\underline[1]{\oldunderline{\smash{#1}}}

\fancyhead[L] {\ifthenelse {\thepage=1}
  {bgs53 (Ben Shulman), sg754 (Spandana Govindgari),\\
   ms969 (MJ Sun), amw275 (Andy Wang)}
  {\hwtype\ \hwnum\ (bgs53, sg754, ms969, amw275)}}
\fancyhead[R] {\ifthenelse {\thepage=1}
  {\classname\\\hwtype\ \hwnum\ (due \duedate)}
  {Page \thepage}}

\begin{document}
\begin{center}\textbf{Project 2: Word Sense Disambiguation -- Proposal}\end{center}

We plan on writing all our code in Python.

\section{Our two WSD systems}

\begin{itemize}
\item \textit{What kinds of features are you planning to extract from the surrounding context for supervised WSD?}\par

STRIP OUT STOP WORDS. Stemming?

We are planning to extract: part-of-speech, tense, plurality, 

take a parameter to go left or right. If negative, take all, otherwise

weighted distance (1/d), or count raised to (1/d), other decay functions for distance.

WSD POS: POS, next-word, next-word, next-word, etc (counts), prev-word, prev-word, etc (counts)

scipy - sparse matrix, nltk

Problem: tons of zeros. BUT NO! The probability of 0 is very very close to 1! YEAH:\par
\[P(``the'' = 4 | class)\]

\item \textit{What are you going to do for finding relevant words in the context for
dictionary-based WSD?}\par
\end{itemize}

STRIP OUT STOP WORDS. Stemming?

Python: import lxml\par
Dictionary has def + examples. Data's in same format. Predict id given pos, gloss, examples. (use ALL of gloss) Examples may be training data. (strip out ``the'', ``and'', etc)

Same approach with prev/next words. Stemming

\textit{Provide a clear and brief explanation of your planned systems and algorithms. (Do not repeat the basic models that are already described in this document) Rather than writing up every single detail, try to explain the motivation of your design decisions by investigating the provided dataset and illustrating the intuition that you discovered from the real examples. Note that better features (typically identified by looking at the training data), the higher accuracies you will likely achieve.}

Preprocess using stemming. Strip out stop words. Feature extraction. Naive: pos, $n$ words around the word, and counts. Weighted regression/averaging.

Dictionary: stemming, stop words. (not in gloss?)

Dictionary: for each unigram/bigram/trigram, find unigram/bigram/trigram in each definition assoc with each class

\section{Implementation schedule}
\textit{Provide a brief implementation schedule.}\par

Division of labor:
\begin{itemize}[noitemsep]
\item MJ: parse data, naive-bayes
\item Ben: naive-bayes
\item Andy: dictionary-based wsd
\item Spandana: dictionary-based wsd
\end{itemize}

Extensions:\par
Do three of four! (Spandana read it on Piazza)

\begin{itemize}[noitemsep]
\item Sat, March 8: most necessary preprocessing is done.
\item Sun, March 9: corrections
\item Tue, March 11: initial data structures for dictionary WSD and supervised wsd underway, some initial testing
\item Fri, March 14: dictionary WSD implementation done, more testing
\item Sun, March 16: supervised WSD implementation done, begin report writeup
\item Mon, March 17: validation and testing for both systems, upload to Kaggle
\item Tue, March 18: any corrections...\ and report to be completed.
\end{itemize}

\end{document}
